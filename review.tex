%\documentclass[draft]{beamer}
\documentclass{beamer}

\usetheme{Warsaw}
\usecolortheme{seahorse}

\usepackage[utf8]{inputenc}
\usepackage[polish]{babel}
\usepackage{polski}
\usepackage{hyperref}
\usepackage{fancyhdr}
\usepackage{verbatim}

\hypersetup{linkbordercolor=1 1 1}

\title{Anomaly Intrusion Detection System}
\author[Michał Bugno \and Antoni Piechnik]
{
  \textbf{Michał Bugno} msq@student.agh.edu.pl \\
  \textbf{Antoni Piechnik} piechnik@student.agh.edu.pl
}
\date{\today}
\begin{document}
\maketitle

\section{Instalacja/użycie}
\subsection{Biblioteki}
\begin{frame}
	\begin{itemize}
	\item PCAP (filtr pakietów)
	\item config (parser plików cfg)
	\item pthread (wątki)
	\item strophe (xmpp)
	\item scons (make w pythonie)
	\item expat (parser xml)
	\item resolv, ssl (zależności)
	\end{itemize}
\end{frame}

\subsection{Jak?}
\begin{frame}
	\begin{itemize}
	\item zbudować biblioteki
	\begin{itemize}
		\item strophe wymaga w folderze \emph{expat} źródeł expat
		\item strophe należy skompilować z obsługą openssl aby obsługiwać np. GTalk (plik \emph{Sconstruct}, TLS\_PLUGIN=``openssl``)
	\end{itemize}
	\item być może ustawić CFLAGS/LDFLAGS ze ścieżkami do plików nagłówkowych/bibliotek
	\end{itemize}
\end{frame}

\begin{frame}
\begin{itemize}
\item \texttt{(sudo?) aids}
\item \texttt{(sudo?) aids -k}
\item aids.cfg, aids.pid
\end{itemize}
\end{frame}

\section{O systemie}
\begin{frame}
System służy do monitorowania obciążenia sieci oraz procesora na maszynach UNIX.
\end{frame}

\end{document}
